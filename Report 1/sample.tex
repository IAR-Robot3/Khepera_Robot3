\documentclass[]{article}

%%%%%%%%%%%%%%%%%%%
% Packages/Macros %
%%%%%%%%%%%%%%%%%%%
\usepackage{amssymb,latexsym,amsmath}     % Standard packages
\usepackage{listings}
\usepackage{algorithm}
\usepackage{algorithmic}

%%%%%%%%%%%
% Margins %
%%%%%%%%%%%
\addtolength{\textwidth}{1.0in}
\addtolength{\textheight}{1.00in}
\addtolength{\evensidemargin}{-0.75in}
\addtolength{\oddsidemargin}{-0.75in}
\addtolength{\topmargin}{-.50in}


%%%%%%%%%%%%%%%%%%%%%%%%%%%%%%
% Theorem/Proof Environments %
%%%%%%%%%%%%%%%%%%%%%%%%%%%%%%
\newtheorem{theorem}{Theorem}
\newenvironment{proof}{\noindent{\bf Proof:}}{$\hfill \Box$ \vspace{10pt}}  


%%%%%%%%%%%%
% Document %
%%%%%%%%%%%%
\begin{document}

\title{IAR Reort\\Task 1}
\author{Babak Esmaeili, David Fullerton}
\maketitle

\begin{abstract}
We have written a program for a Khepera robot that allows the robot to following along a wall while avoiding colliding with obstacles. When the robot is first released into the arena it will drive until it detects a surface, if the obstacle is detected with the front two IR sensors the robot will turn counter-clockwise on the spot until its right side if facing the wall. However if the robot first encounters an object on its left side, it will curve away to the right. Once the robot detects an object using sensor 6, the sensor on its right side, the robot follows the surface of the object attempting to keep within a set distance away from it, we use PID controls for this. If the robot is placed in an empty arena this will result in the robot driving until it reaches the wall and then following the wall in a counter-clockwise direction. Some limitations with our design are that the robot only uses one sensor to measure the distance from the wall when using PID control also if the robot meets a small object before a wall it will follow the edge of that object.
\end{abstract}


\section{Introduction}
%%%%%%%%%%%%%%%
The task was to design a controller for a Khepera robot that would navigate an arena while avoiding collisions with obstacles. If possible the robot would tend to follow long walls and avoid following the edge of smaller obstacles. We approached this by deciding the robot would favour keeping an object a set distance away from its right hand side. As for how the robot would decide if what it sensed was a wall or small object we were unsure how to approach this. One idea was that when a sensor received a reading above a threshold we would also check the adjacent sensors as to what their values were. If they were also above a threshold then it would suggest a wall, if not then a small object. We didn’t use this as it would only work for very small objects and for the front middle four sensors. We were late to start this project and so referred to a previous course, IVR, and course work we had done in the past to help our approach to this task.


\section{Methods}
%%%%%%%%%%%%%%%%%%%
First, we had decide the value for the distance threshold, which would determine robots closeness to the wall. The most important IR sensor in our case is number 6, which is the sensor that is used to measure the closeness to the wall. The robot’s distance threshold is set the threshold to 150. If it detects a wall (or an obstacle) close to it, it will go to PID control in order to maintain the distance threshold to the wall.\\

\begin{center}
\begin{tabular}{ |c| } 
 \hline
 \textbf{Main Controller Algorithm} \\
 \hline
 \\
 If haven't detected the wall $=>$ Keep going Straight \\ \\
 If reached a corner $=>$ Turn left on the spot \\ \\
 If detecting an obstacle on the left side $=>$ Curve to the right \\ \\
 Else go to PID control and maintain the distance to the wall \\ \\
 \hline
\end{tabular}
\end{center}

\begin{algorithm}                      % enter the algorithm environment
\caption{PID Controller}          % give the algorithm a caption
\label{alg1}                           % and a label for \ref{} commands later in the document
\begin{algorithmic}                    % enter the algorithmic environment

    \WHILE{$N \neq 0$}
        \IF{Obstacle on left detected}
        
            \STATE $X \Leftarrow X \times X$
            \STATE $N \Leftarrow N / 2$
        \ELSE[$N$ is odd]
            \STATE $y \Leftarrow y \times X$
            \STATE $N \Leftarrow N - 1$
        \ENDIF
    \ENDWHILE
\end{algorithmic}
\end{algorithm}

One challenge we faced during the experiments was the buffering problem. Due to some power problems and possibly sending too many commands, our robot crashes. The reason for that is sending unwanted command to the robot but without printing it or in other words keeping the buffer clean. First we made sure that whenever we are about to set the speed wheels same as the previous loop, we avoid sending that command again and simply go to the next loop. We are also pausing the program for 0.1 s at the end of the loop to make sure there is enough time spaces between commands.    

\section{Results}
%%%%%%%%%%%%%%%%
Our robot is able to follow the wall properly and avoid any obstacles both on its left side and right side. In trials, each trial including on full round with some obstacle in the way, it got stuck n times, next to a grey object. We observed that it detects the darker objects, much later than the lighter objects. In other words, it must be very close to the dark object in order to detect it, which is the moment that there is sudden huge change in the sensor values. The two main problems it faced during problems was the occasional crashing, and wiggling too much around grey or dark objects due to the delay in detecting them.\\\\

We managed to reduce the crashing by the method described in the previous section. The reason for the struggle in following dark walls was the PID control. In the first approach only PD control was used, as we observed the robot doing a reasonable job. In order to reduce the wiggling with the grey objects, we increased the constant multipliers for D and I parameter. \\
We have experimented obstacle avoidance and wall following using both “Turning on the spot” method and “Curved path” method. We observed that the using curved path performs quite better than turning right at the spot. Using curved turning, our robot only got syck n times, while using “turning on the spot” method, it got stuck m times. In general, curved paths are better than turning on the spot, as a fluid motion is more desirable and turning on the spot slows down the movement.

\section{Discussion}
%%%%%%%%%%%%%%%%%%%
One improvement that can be done is giving the robot a side to follow. At the moment, it always follows the wall clockwise, even if the initial situation is one where following anti-clockwise is easier and more sensible. We can turn the other way around by using IR sensor 1 instead of IR sensor 6. We can make side an argument, or even better. we can compare the sensor values from both sides and check which side the wall is and then follow the way which is more sensible. \\\\

If the robot is positioned in the center of the arena (with some obstacles in the middle), Ideally, if it detects an obstacle, it should go around it and then go straight until it reaches the wall and then following the wall. However, what occurs at the moments is the robot circling that obstacle and treating it as the wall. In other words, the robot cannot differentiate between an obstacle and the wall. One possible solution to overcome this problem is odometry. By tracking its position, if the robot is reaching the same position in a short amount of time, it can conclude that the object next to it is an obstacle and not the wall. Thus, it can keep going straight after one turn until it finds the wall.

\section{Appendix}
%%%%%%%%%%%%%%%%%%%%
\lstinputlisting[language=bash]{robot3_controller.m}

\end{document}